\chapter{Conclusione}
Il lavoro descritto in questa tesi è consistito nello sviluppo e integrazione di un sensore Bluetooth all'interno dell'applicazione GeneroCity, con l'obiettivo di rilevare in modo affidabile quando un utente è alla guida di un veicolo. Questo modulo software, congiuntamente agli altri sensori, contribuirà a rendere l'esperienza degli utenti più sicura e comoda, riducendo la necessità di interazioni manuali durante la guida, facilitando così lo scambio di parcheggi con altri utenti che utilizzano l'applicazione. L'implementazione del sensore ha richiesto un'analisi approfondita delle API Android e la creazione di una strategia per riconoscere la connessione degli smartphone con delle automobili. Infine il sensore sviluppato ha dimostrato di essere efficace nei test effettuati, rilevando correttamente la connessione a dispositivi compatibili con il contesto di guida, come le autoradio.


\section{Sviluppi futuri}
Per migliorare ulteriormente il sistema, sono attualmente in sviluppo presso il Gamification Lab modelli di machine learning per il calcolo dello stato dell'utente, così da sostituire in futuro la media pesata della confidenza dei vari sensori con algoritmi più sofisticati. Il sensore da me sviluppato, oltre agli scopi descritti precedentemente, verrà utilizzato insieme agli altri sensori anche per raccogliere dati utili all'allenamento di questi modelli. Inoltre, sarebbe utile implementare un servizio esterno per verificare se il nome del dispositivo Bluetooth corrisponda a modelli di veicoli conosciuti, incrementando così l'accuratezza del riconoscimento e adattandosi a veicoli nuovi. Questo servizio potrebbe essere implementato ad esempio attraverso un nuovo endpoint esposto dalla web API di GeneroCity. In questa maniera sia l'applicazione Android sia quella iOS potrebbero farne uso, in modo da effettuare in modo centralizzato questa computazione.

L'utilizzo del sistema di rilevamento basato su sensori intelligenti come quello descritto in questa tesi per ottenere interazioni implicite rappresenta una direzione promettente per lo sviluppo di GeneroCity e più in generale nel campo della mobilità urbana.

