\chapter{Conclusione}
Il lavoro descritto in questa tesi ha portato allo sviluppo e all'integrazione di un sensore Bluetooth all'interno dell'applicazione GeneroCity, con l'obiettivo di rilevare in modo affidabile quando un utente è alla guida di un veicolo. Questa soluzione, congiuntamente agli altri sensori, contribuisce a rendere l'esperienza degli utenti più sicura e comoda, riducendo la necessità di interazioni manuali durante la guida e facilitando la gestione intelligente dei parcheggi urbani. L'implementazione del sensore ha richiesto un'analisi approfondita delle API Android e la creazione di una strategia per la gestione delle connessioni Bluetooth. Infine il sensore sviluppato ha dimostrato di essere efficace nei test effettuati, rilevando correttamente la connessione a dispositivi compatibili con il contesto di guida, come le autoradio.


\section{Sviluppi futuri}
Per migliorare ulteriormente il sistema, sono attualmente in sviluppo presso il Gamification Lab modelli di machine learning per il calcolo dello stato di guida, così da sostituire in futuro la media pesata della confidenza dei vari sensori con algoritmi più sofisticati. Inoltre, sarebbe utile implementare un servizio esterno per verificare se il nome del dispositivo Bluetooth corrisponda a modelli di veicoli conosciuti, incrementando così l'accuratezza del riconoscimento. Questo servizio potrebbe essere implementato ad esempio attraverso un nuovo endpoint esposto dalla web API di GeneroCity, in questa maniera sia l'applicazione Android sia quella iOS potrebbero farne uso, in modo da effettuare in maniera centralizzata questa computazione.

L'utilizzo del sistema di rilevamento basato su sensori intelligenti come quello descritto in questa tesi per ottenere interazioni implicite rappresentano una direzione promettente per lo sviluppo di GeneroCity e più in generale nel campo della mobilità urbana.

