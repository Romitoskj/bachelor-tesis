\begin{abstract}
L'elaborato descrive lo sviluppo e l'integrazione di un sistema per rilevare gli spostamenti in auto tramite il Bluetooth degli smartphone nell'applicazione GeneroCity, un'applicazione di smart parking progettata dal Gamification Lab della Sapienza Università di Roma. Questo sistema, chiamato sensore Bluetooth, è in grado di rilevare automaticamente quando l'utente è alla guida di un'auto, sfruttando le interazioni implicite e analisi contestuale dei dati forniti dal Bluetooth degli smartphone, ed è stato sviluppato utilizzando le API di Android per la gestione delle connessioni Bluetooth. Il sistema è in grado di riconoscere se l'utente è alla guida con una certa confidenza contribuendo al riconoscimento dello stato dell'utente tramite un algoritmo di media pesata tra vari sensori. I test effettuati hanno dimostrato l'affidabilità del sensore nello scenario reale, confermandone l'efficacia e la corretta integrazione con l'app GeneroCity. Futuri sviluppi prevedono l'introduzione di modelli di machine learning per migliorare la determinazione dello stato dell'utente e l'implementazione di servizi esterni per il riconoscimento dei dispositivi Bluetooth.
\end{abstract}