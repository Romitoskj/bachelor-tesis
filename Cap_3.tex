\chapter{I Sensori e il loro utilizzo per ottenere Interazioni Implicite}

Come detto in precedenza una delle funzionalità principali dell'applicazione è riconoscere come l'utente si sta spostando in un determinato momento e, in questa fase dello sviluppo, ci siamo concentrati nel costruire un sistema che deducesse in modo autonome se un'utente si trova alla guida o meno. Di seguito si spiegherà come questo scopo è stato raggiunto attraverso il lavoro coordinato di vari sensori.


\section{I sensori}
In GeneroCity un sensore è un modulo software\footnote{La programmazione modulare è un paradigma di programmazione che consiste nella realizzazione di programmi suddivisi in moduli, ognuno dei quali svolge precise funzioni.} che analizza il contesto in cui si trova l'utente in uno specifico istante per determinare l'azione compiuta da quest'ultimo. In particolare ciascuno di essi calcola un valore reale compreso tra 0 e 1, detto \textbf{confidenza}, il quale rappresenta il grado di sicurezza con il quale il sensore ha effettuato la rilevazione del'azione (dove 0 rappresenta una sicurezza minima e 1 massima). Poiché nella versione attuale dell'applicazione l'azione che viene rilevata è solamente una, possiamo approssimare la confidenza come la probabilità che l'utente stia guidando, più specificamente:
\begin{itemize}
    \item un valore compreso tra 0 e 0,5 (stato \textit{walking}) indica che l'utente non sta guidando;
    \item il valore 0,5 (stato \textit{unknown}) denota che il sensore non è in grado di inferire lo stato dell'utente;
    \item una confidenza compresa tra 0,5 e 1 (stato \textit{automotive}) segnala che l'utente sta guidando.
\end{itemize}

Per analizzare il contesto e calcolare la confidenza ogni sensore può adottare svariati approcci: esso si può basare sulle rilevazioni di un vero e proprio sensore (come quelli di movimento o ambientali), sullo stato di una specifica componente del dispositivo, ad esempio la batteria o il display, oppure può far uso della connettività Wi-Fi, Bluetooth o utilizzare altre tecnologie e protocolli specifici, come la geolocalizzazione attraverso GPS. Una linea guida importante per lo sviluppo di un sensore è che esso si avvalga una sola di queste tecnologie o componenti in modo da effettuare la computazione in maniera indipendente dagli altri: ad esempio non succederà mai che il sensore Bluetooth effettui una scansione dei dispositivi vicini quando il GPS rileva che l'utente si sta muovendo ad una certa velocità. Questo perché sarà l'insieme dei risultati ottenuti da tutti i sensori a concorrere al calcolo dell'effettivo stato dell'utente ed eventuali correlazioni tra sensori differenti verranno prese in considerazione dal sistema durante questa computazione tramite apprendimento automatico.


\section{Il flusso di aggiornamento della confidenza in GeneroCity Android}
Dato che i sensori sono molteplici un fattore importante è la loro coordinazione: lo stato dell'utente sarà determinato basandosi sulla confidenza di ogni sensore ed è quindi importante che esso sia aggiornato ogni qualvolta un sensore cambia la sua confidenza. Per sincronizzare il lavoro dei sensori è stato quindi fondamentale delle funzionalità comuni che ogni sensore deve avere. Questo è stato possibile sfruttando il paradigma di programmazione orientata agli oggetti implementato in Java\cite{Java}, in particolare è stata definita una classe astratta\footnote{Nella programmazione orientata agli oggetti una classe astratta è una classe che non può essere istanziata, la quale definisce delle funzionalità di base per tutte le sue sotto classi.} che tutti i sensori ereditano.

\subsection{La classe astratta Sensor}
Ogni sensore eredita dalla classe Sensor i seguenti attributi: il \textit{nome} univoco che identifica il sensore, la \textit{versione} del sensore, il \textit{peso} (rappresentato come numero reale) che ha il sensore nel calcolo dello stato dell'utente.

Inoltre ogni sensore mantiene uno storico della confidenza che ha calcolato tramite una mappa ad albero\footnote{Una treemap è una struttura dati che implementa le funzionalità di normale una normale mappa, ossia immagazzina delle coppie (chiave, valore), e inoltre mantiene l'ordinamento delle coppie basato sull'ordinamento naturale delle chiavi.} con chiave lo Unix Timestamp\footnote{Lo Unix Timestamp rappresenta il numero di secondi trascorsi dalla mezzanotte del 1° Gennaio 1970.} del momento in cui il valore è stato calcolato.

Invece i principali metodi esposti dai sensori invece sono:
\begin{itemize}
    \item \textit{getStatus} che ha come parametro un Timestamp e restituisce la confidenza del sensore calcolata nell'istante più vicino a quello passato come parametro. Questo metodo è l'unico metodo astratto ed è quindi il metodo principale che i sensori dovranno implementare.
    \item \textit{update} che riceve in input un Timestamp e un oggetto contenente i dati inviati dal sensore. Questo metodo si occuperà di inviare i dati raccolti al server e di innescare il ricalcolo dello stato dell'utente.
\end{itemize}
Quando viene eseguito il metodo update viene notificata l'unità centrale (una classe statica chiamata SensorConstants) che richiede lo stato di ogni sensore (utilizzando il metodo \textit{getStatus}) e calcola il nuovo stato dell'utente. Attualmente questo stato viene rappresentato come la media pesata delle confidenza di ogni sensore in uno specifico momento.

\subsection{L'invio di dati al server}
Come anticipato ogni qualvolta viene aggiornata la confidenza di un sensore il metodo update richiede in input un oggetto che rappresenta lo stato del sensore al momento dell'aggiornamento (ad esempio per il sensore Wi-Fi i dati della rete a cui è connesso, mentre per il sensore GPS le coordinate geografiche in cui l'utente si trova). 

\section{Il compito del sensore Bluetooth}
L'obbiettivo del mio tirocinio: sviluppare un sensore per capire se l'utente sta guidando basandosi sui dispositivi connessi al bluetooth
Scelta di implementarlo prima in un progetto separato, motivazioni:
\begin{itemize}
    \item prendere dimestichezza con l'SDK di Android
    \item imparare ad utilizzare le API di Android per la connettività via bluetooth per
    \begin{itemize}
        \item rilevare l'accensione e lo spegnimento del bluetooth stesso
        \item ottenere i dati dei dispositivi connessi
        \item effettuare scansioni per trovare dispositivi vicini
    \end{itemize}
\end{itemize}
